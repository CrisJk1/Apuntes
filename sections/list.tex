\section{Contenedor \texttt{list}}

\subsection{Introducción}
\texttt{list} es un contenedor secuencial que implementa una \textbf{lista doblemente enlazada}.  
Cada elemento contiene punteros al anterior y al siguiente, lo que permite insertar o eliminar elementos en cualquier posicion en tiempo constante O(1), siempre que se tenga un iterador a esa posicion.  
A diferencia de \texttt{vector}, no ofrece acceso aleatorio rapido (no se puede usar el operador \texttt{[]}).

\subsection{Operaciones importantes}

\subsubsection{Declaracion e inicializacion}
\begin{lstlisting}
#include <list>
#include <string>

list<int> l1;                 // vacia
list<int> l2(5, 100);         // 5 elementos con valor 100
list<int> l3 = {1, 2, 3, 4};  // inicializacion con lista
list<string> names = {"Ana", "Luis"};
\end{lstlisting}

\subsubsection{Agregar y eliminar elementos al inicio o final}
\texttt{push\_front()} y \texttt{push\_back()} permiten insertar rapido al inicio o final.  
\texttt{pop\_front()} y \texttt{pop\_back()} eliminan esos elementos.
\begin{lstlisting}
list<int> l = {2, 3};
l.push_front(1);   // {1,2,3}
l.push_back(4);    // {1,2,3,4}
l.pop_front();     // {2,3,4}
l.pop_back();      // {2,3}
\end{lstlisting}

\subsubsection{Acceder al primer y ultimo elemento}
\begin{lstlisting}
if (!l.empty()) {
    cout << "first: " << l.front()
              << ", last: " << l.back();
}
\end{lstlisting}

\subsubsection{Insertar y borrar en posiciones arbitrarias}
Usando iteradores, se puede insertar o eliminar en cualquier parte en O(1).
\begin{lstlisting}
list<int> a = {1,2,4};
auto it = next(a.begin(), 2); // apunta al 4
a.insert(it, 3);                   // {1,2,3,4}
a.erase(next(a.begin()));     // elimina el 2 -> {1,3,4}
\end{lstlisting}

\subsubsection{Recorrer la lista}
\begin{lstlisting}
// range-based for
for (int x : a) cout << x << ' ';

// iteradores
for (auto it = a.begin(); it != a.end(); ++it)
    *it *= 2;
\end{lstlisting}

\subsubsection{Tamaño y estado}
\begin{lstlisting}
cout << "size: " << a.size() << "\n";
if (a.empty()) cout << "lista vacia\n";
\end{lstlisting}

\subsubsection{Eliminar elementos especificos}
\begin{lstlisting}
list<int> b = {1,2,3,2,4};
b.remove(2); // elimina todos los valores iguales a 2 -> {1,3,4}
\end{lstlisting}

\subsubsection{Ordenar, invertir y eliminar duplicados}
\texttt{sort()}, \texttt{reverse()} y \texttt{unique()} operan en la propia lista.
\begin{lstlisting}
list<int> nums = {4,1,3,2,3};
nums.sort();     // {1,2,3,3,4}
nums.unique();   // {1,2,3,4}
nums.reverse();  // {4,3,2,1}
\end{lstlisting}

\subsubsection{Fusionar dos listas ordenadas}
\begin{lstlisting}
list<int> l1 = {1,3,5};
list<int> l2 = {2,4,6};
l1.merge(l2); // l1 = {1,2,3,4,5,6}, l2 queda vacia
\end{lstlisting}

\subsubsection{Intercambiar contenidos}
\begin{lstlisting}
list<int> x = {1,2}, y = {9,8};
x.swap(y); // x = {9,8}, y = {1,2}
\end{lstlisting}

\subsection{Complejidad tipica}
\begin{itemize}
  \item Acceso secuencial: O(n)
  \item Insercion/eliminacion con iterador: O(1)
  \item Borrado por valor (\texttt{remove}): O(n)
  \item Ordenar (\texttt{sort}): O(n log n)
\end{itemize}