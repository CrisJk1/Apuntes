\section{Adaptador \texttt{stack}}

\subsection{Introducción}
\texttt{stack} es un adaptador de contenedor que implementa una \textbf{pila (LIFO - Last In, First Out)}.
Utiliza otro contenedor subyacente (por defecto \texttt{deque}, pero puede ser \texttt{vector} o \texttt{list}) y restringe el acceso únicamente al elemento superior.
Es ideal cuando se necesita procesar elementos en orden inverso al de inserción.

\subsection{Operaciones importantes}

\subsubsection{Declaración e inicialización}
\begin{lstlisting}
#include <stack>

stack<int> s1;                      // pila vacia (usa deque)
stack<int, vector<int>> s2;         // pila con vector subyacente
stack<int, list<int>> s3;           // pila con list subyacente
\end{lstlisting}

\subsubsection{Agregar elemento: \texttt{push()}}
Añade un elemento al tope de la pila. Complejidad O(1).
\begin{lstlisting}
stack<int> s;
s.push(10);
s.push(20);
s.push(30); // tope: 30
\end{lstlisting}

\subsubsection{Acceder al elemento superior: \texttt{top()}}
Devuelve una referencia al elemento en el tope (no lo elimina).
\begin{lstlisting}
if (!s.empty()) {
    cout << "Tope: " << s.top() << "\n"; // 30
    s.top() = 99; // se puede modificar el tope
}
\end{lstlisting}

\subsubsection{Eliminar elemento superior: \texttt{pop()}}
Elimina el elemento del tope. No devuelve su valor.
\begin{lstlisting}
if (!s.empty()) {
    int valor = s.top(); // primero obtener el valor
    s.pop();             // luego eliminar
}
\end{lstlisting}

\subsubsection{Tamaño y estado}
\begin{lstlisting}
cout << "size: " << s.size() << "\n";
if (s.empty()) {
    cout << "pila vacia\n";
}
\end{lstlisting}

\subsubsection{Patrón típico: procesar todos los elementos}
\begin{lstlisting}
while (!s.empty()) {
    int valor = s.top();
    s.pop();
    cout << valor << " "; // procesar en orden LIFO
}
\end{lstlisting}

\subsubsection{Intercambiar contenidos}
\begin{lstlisting}
stack<int> x, y;
x.push(1); x.push(2);
y.push(9);
x.swap(y); // x tiene {9}, y tiene {1,2}
\end{lstlisting}

\subsection{Complejidad típica}
\begin{itemize}
  \item \texttt{push}: O(1)
  \item \texttt{pop}: O(1)
  \item \texttt{top}: O(1)
  \item No permite iteración directa ni acceso aleatorio
\end{itemize}
