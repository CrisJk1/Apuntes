\newpage
\section*{Ejemplos}

\subsection*{Fuerza Bruta}

1)Prefijo Primo: Se busca contar la cantidad de números de tres dígitos, de forma que
cumplen que cada prefijo es también un número primo.\\

\begin{lstlisting}
#include <bits/stdc++.h>
using namespace std;

//Algorithm isItPrime implmented with brute force O(n)
bool isItPrime(int number){
    if(number <= 1)
        return false;
    
    for(int i = 2 ; i < number ; i++){
        if( number%i == 0) 
            return false;
    }

    return true;
    
}
//Algorithm analizeThree implmented with brute force O(n**3)
void analizeThree(){
    int counter = 0;
    for (int i = 1 ; i < 10 ; i++){
        for (int j = 0 ; j < 10 ; j++){
            for (int k = 0 ; k < 10 ; k++){
                int second = (i * 10) + j;
                int third = (i * 100) + (j * 10) + k;
                if( isItPrime(i) && isItPrime(second) && isItPrime(third)) 
                    counter ++;
            }
        }
    }
    cout << counter;
}

int main(){
    analizeThree();
    return 0;
}
\end{lstlisting}

\newpage

2) Imprima todos los subconjuntos de un conjunto dado, por ejemplo:
\begin{multicols}{2}

\begin{center}
    Input
\end{center}

\begin{lstlisting}
3 \\tamano del conjunto
1 2 3 \\conjunto
\end{lstlisting}

\newcolumn

\begin{center}
    Output
\end{center}

\begin{lstlisting}
{}
{1}
{2}
{3}
{1, 2}
{1, 3}
{2, 3}
{1, 2, 3}
\end{lstlisting}

\end{multicols}



\begin{lstlisting}
#include <bits/stdc++.h>
using namespace std;

void analizeSet(vector<int> set, int n){
    int totalSubSets = 1 << n;
    for (int i=0; i < totalSubSets; i++){
        cout << "{";
        //logica es esta
        for (int j= 0; j < n; j++){
            if ( i & ( 1<<j )){
                cout << set[j];
                if (j < n-1) cout << ", ";
            }
        }
        cout << "}"<< endl;
    }

}

void getSet(){
    int n;
    cin >> n;
    vector<int> set(n);
    for (int i = 0 ; i < n ; i++){
        cin >> set[i];
    }

    analizeSet(set, n);
}

int main(){
    getSet();
    return 0;
}
\end{lstlisting}