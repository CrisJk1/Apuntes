\section{Funciones Útiles y Patrones Comunes}

\subsection{Generación de Permutaciones}

\subsubsection{Función para generar todas las permutaciones}
Utiliza \texttt{next\_permutation()} para generar todas las permutaciones de un vector.
El vector debe estar inicialmente ordenado. Complejidad: O(n! * n).
\begin{lstlisting}
void perm(vector<char> &arr){
    do{
        for(auto c : arr){
            cout << c << " ";
        }
        cout << endl;
    } while(next_permutation(arr.begin(), arr.end()));
}
\end{lstlisting}

\subsection{Generación de Subconjuntos (Bitmask)}

\subsubsection{Función para generar todos los subconjuntos}
Utiliza máscaras de bits para generar todos los subconjuntos posibles (conjunto potencia).
Para un conjunto de tamaño n, genera $2^n$ subconjuntos. Complejidad: O($2^n$ * n).
\begin{lstlisting}
void mask(vector<char> &arr){
    int n = arr.size();
    int total_subset = 1 << n;  // 2^n subconjuntos

    for(int i = 0; i < total_subset; i++){
        for(int j = 0; j < n; j++){
            if(i & (1 << j)){
                cout << arr[j];
                if(j < n - 1) cout << " ";
            }
        }
        cout << endl;
    }
}
\end{lstlisting}